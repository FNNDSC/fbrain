\section{The whole pipeline}
BTK allows to implement the whole pipeline for the processing of fetal images,
i.e. the reconstruction of anatomical and diffusion data, and the final
tractography, all expressed in the same local coordinate system. This
processing can be summarized in the following steps:

\begin{enumerate}
 \item Image conversion
 \item Anatomical image reconstruction
 \item Recontruction of the diffusion sequence
 \item Registration of diffusion to anatomical data
 \item Tractography
\end{enumerate}

In the following we explain the applications and utilities to use for each of
the aforementioned steps.

\subsection{Image conversion}
BTK supports and has been tested by using images in Nifti format
(http://nifti.nimh.nih.gov/nifti-1). However, images are frequently available in
DICOM format and an image conversion is required. This can be performed by using
dcm2nii, a converter freely available at http://www.cabiatl.com/mricro/mricron/
dcm2nii.html. In this site you can find other options that can be best suited
for your data.

\subsection{Anatomical image reconstruction}
This can be performed by using \textbf{btkImageReconstruction} (Section
\ref{subsec:ana_rec}) followed by \textbf{btkReorientImageToStandard} (Section
\ref{sec:utilities}) to transform the result to a standard orientation.

\subsection{Reconstruction of the diffusion sequence}
To reconstruct diffusion data, you want to follow the steps described in
Section \ref{subsec:diff_rec}. The use of two applications is required here:
\textbf{btkGroupwiseS2SDistortionCorrection} and
\textbf{btkRBFInterpolationS2S}.

\subsection{Registration of diffusion to anatomical data}
TO BE WRITTEN

\subsection{Tractography}
If you have followed the previous steps correctly, at this point you should
have the reconstructed anatomical and diffusion data spatially aligned, and
ready to perform the tractography. To do this, BTK provides
\textbf{btkTractography} (Section \ref{subsec:tracto})
